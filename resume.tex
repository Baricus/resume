%=======================================
% resume.tex - a transition to typsetting in LaTeX
%
% Written by Miles Shamo
%	
%	My main goal in transitioning to LaTeX (LuaLaTeX techincally)
%	is to have better control over element layout, as I'm through
%	with dealing with MS Word textboxes.  At the same time,
%	I decided to upload this to github, to have better version control
%	for individual opportunities
%	
%	To Do:
%		- Redo Relevant coursework
%		- Add vector work:
%			-bullets for education, skills, etc?
%		- Convert to environments
%
%	Considerations:
%		- Should relevant coursework be revamped to entirely technical electives?
%		- Github Actions based releases rather than included PDF?


%------------------------------------------------------------------------------PACKAGES/SETUP
%page geometry setup
\documentclass[10.5pt, letterpaper]{article}
\usepackage[letterpaper, includeheadfoot, top=0.25in, bottom=0.25in, left=0.5in, right=0.5in, headheight=50pt, headsep=8pt]{geometry}

%font setup
\usepackage{fontspec}
\setmainfont[Ligatures=TeX]{Garamond}

%various other imports
\usepackage{fancyhdr}          % headerst
\usepackage{xcolor}            % color
\usepackage{hyperref}          % hyperlink reference
\usepackage{microtype}         % font resizing
\usepackage{tabularx}          % for expandable columns
\usepackage{enumitem}          % better lists
\usepackage{multicol}          % multiple columns
\usepackage[compact]{titlesec} % allows setting title spacing


%hyperlink setup to better match standard
\hypersetup{final=true, colorlinks, urlcolor=blue}

%sets up rows for tabular eviornment
\setlength{\extrarowheight}{0pt}
\newcolumntype{E}{>{\raggedleft\arraybackslash}X} % Education columns (left side alignment)
\newcolumntype{Y}{>{\hsize=.11\hsize}X}           % Year columns (fixed size, center aligned)

% tabularX removing spacing
\renewcommand\tabularxcolumn[1]{p{#1}}
\renewcommand*{\arraystretch}{0}

%sets up descriptions to be dense and off the next line
\setlist[description]{topsep=0pt, itemsep=2pt, parsep=0pt, style=nextline}

% vertical spacing setups
% lowes spacing of section headers (left spacing, above, below)
\titlespacing{\section}{0pt}{-10pt plus 0pt minus 2pt}{0pt plus 0pt minus 0pt}
% reduces multicol spacing
\setlength{\multicolsep}{0pt plus 0pt minus 0pt}
% reduces tabularx prespacing
\setlength{\textfloatsep}{0pt}
% removes space before front of lists
\setlength\topsep{0pt}
% lowers seperation between list items
\setlength\itemsep{3pt}

% ---------------------------------------------------------------------------Custom environments
% simple environments for the start/end of various sections
% These is largely just a section header and a flush left for formatting
\newenvironment{Education} %
{                          %
	\section*{Education}   %
	\flushleft             % ensures tables work by forcing alignment
}{                         %
	\endflushleft          %
}

\newenvironment{Experience} %
{                           %
	\section*{Experience}   %
	\flushleft              %
}{                          %
	\endflushleft           %
}

% Activites are formatted into a multicolumn setup without anything fancy

\newenvironment{Activities}          %
{                                    %
	\section*{Additional Activities} %
	\begin{center}                   %
		\begin{multicols}{2}         %
			\begin{description}      %
}{                                   %
			\end{description}        %
		\end{multicols}              %
	\end{center}                     %
}

% The following three environments use tabularx to fix things into a two column format
% which is fully controllable
% Thus, to swap between the columns, the & is used as if it was a normal tabularx environment

\newenvironment{Coursework}        %
{                                  %
	\section*{Relevant Coursework} %
	\flushleft                     %
	\tabularx{\textwidth}[c]{X X}  %
}{                                 %
	\endtabularx                   %
	\endflushleft                  %
}

\newenvironment{Projects}         %
{                                 %
	\section*{Projects}           %
	\flushleft                    %
	\tabularx{\textwidth}[c]{X X} %
}{                                %
	\endtabularx                  %
	\endflushleft                 %
}

\newenvironment{Skills}           %
{                                 %
	\section*{Skills}             %
	\flushleft                    %
	\tabularx{\textwidth}[c]{X X} %
}{                                %
	\endtabularx                  %
	\endflushleft                 %
}

% TitleDescription - a "single title" description
% Takes in a title to wrap in square quotes on the first call to \item,
% rebinding \item to work transparently with this as if there was genuinely a title
%
% Arguments: {Description Title}
\newenvironment{TitleDescription}[1]                    %
{                                                       %
	\begin{description}                                 %
		\item [#1]                                      %
		\let\realitem\item                              % rebind \item for first usage in environment
		\renewcommand{\item}[1]{\let\item\realitem ##1} % after first usage, reset it to normal
}{                                                      %
	\end{description}                                   %
}

%EdEntry - Formats a single "school"
%
% Arguments: {School Name}
%			 {Starting Month + Year}
%			 {Ending Month + Year}
%
% Uses \item for each line of description regarding the school.
% 
% NOTE: Due to tabularx, this doesn't seem like it can be nested in further \newenvironment commands
%       Furthermore, environments can't span table cells so we have one table per entry

\newenvironment{EdEntry}[3]                 %
{                                           %
	\newcommand{\DateLine}{#2 \\ to \\ #3}  % stores the date since args aren't present at end
	\tabularx{\textwidth}[c]{E Y}           % fixed width date column
		\begin{multicols}{2}                % 2 columns for education description
			\begin{TitleDescription}{#1}    %
}{                                          %
			\end{TitleDescription}          %
		\end{multicols}                     %
		&                                   % swaps to the date column
		\begin{center}\DateLine\end{center} %
		\\                                  %
	\endtabularx                            %
}

% ExpEntry - formats one experience/job
%
% Arguments: {Position Title}
%			 {Starting Month + Year}
%			 {Ending Month + Year}
%
% Uses \item for each line of description
%
% Very similar to EdEntry, minus the multicols
\newenvironment{ExpEntry}[3]               %
{                                          %
	\newcommand{\DateLine}{#2 \\ to \\ #3} %
	\tabularx{\textwidth}[c]{E Y}          %
		\begin{TitleDescription}{#1}       %
}{                                         %
		\end{TitleDescription}             %
	&                                      % swaps to the date column
	\begin{center} \DateLine \end{center}  %
	\\                                     %
	\endtabularx                           %
}



%Header setup
\pagestyle{fancy}
\fancyhf{}
\lhead
{
	{\Huge \bf Miles Shamo}                                                   \\
	{\Large \emph{Expected Bachelors of Science in Computer Science in 2023}}
}
\rhead
{
	Naperville, IL                                                            \\
	630-465-8997                                                              \\
	\href{mailto:milesshamo@gmail.com}{milesshamo@gmail.com}
}

%---------------------------------------------------------------------------DOCUMENT
\begin{document}

%---------------------------------------------------------------------------Education
\begin{Education}
	\begin{EdEntry}{University of Illinois at Chicago, Chicago, Il.}{Aug. 2019}{May 2023}
		\item B.S. \& M.S. in Computer Science 
		\item 4.0 grade point average
		\item Honors College student
		\item Member of the A.C.M.
	\end{EdEntry}
\end{Education}

%---------------------------------------------------------------------------Coursework
\begin{Coursework}
		\begin{description}
			\item Data Structures (CS 251)
			\item Software Design (CS 342)
			\item Computer Algorithms (CS 401)
		\end{description} &
		\begin{description}
			\item Formal Logic (MATH 430)
			\item Differential Equations (MATH 220)
			\item Linear Algebra (MATH 320)
		\end{description}
\end{Coursework}

%---------------------------------------------------------------------------Skills
\begin{Skills}
	%other skills
	\begin{description}
		\item Various Linux command line utilities
		\item Wireless networking analysis
		\item Regular expressions
		\item GNUMake, Valgrind, and GDB
		\item Computer maintenance and repair	
		%\item CAD software
	\end{description} &

	%Programming languages
	\begin{TitleDescription}{Programming Languages}
		\item Exploring Haskell and Coq
		\item Experienced with C \& C++
		\item Proficient in Bash, Python \& \href{https://github.com/baricus/resume}{\LaTeX{}}
		\item Familiar with JavaScript, Java, AHK, \& Rockstar
	\end{TitleDescription} 
\end{Skills}

%---------------------------------------------------------------------------Experience (work)
%For this section, we return to the formatting of Education
\begin{Experience}
	% Telephony Studies CACI 2022
	\begin{ExpEntry}{Telephone Studies Intern, CACI}{May 2022}{Aug. 2022}
		\item Explored the effectiveness of honeypots as a method of collecting information with a focus on SSH attacks
		\item Analyzed binaries, network traffic, and other attacker activities to garner insights into individual attackers and trends
	\end{ExpEntry}

	% Telephony Studies CACI 2021
	\begin{ExpEntry}{Telephone Studies Intern, CACI}{May 2021}{Aug. 2021}
			\item Designed and developed a hypervisor-side anti-malware solution for guests utilizing virtual machine introspection
			\item Worked extensively with the Linux kernel to identify and test attack vectors
	\end{ExpEntry}

	%251 TA, UIC
	\begin{ExpEntry}{CS 251 (Data Structures) Teaching Assistant, University of Illinois at Chicago}
		{Aug. 2020}{May 2021}
			\item Engaged with students during self-run office hours, lab sections and as an oral exam proctor
			\item Provided iterative feedback on project design and requirements
	\end{ExpEntry}

	%MAKERSPACE, UIC
	\begin{ExpEntry}{Intern, University of Illinois at Chicago MakerSpace}
		{July 2020}{Aug. 2020}
		\item Independently designed and manufactured a desk-mounted microphone boom arm using MakerSpace facilities
		\item Designed and manufactured sneeze guards for use in UIC offices
		%\item Experimented with various additive and subtractive manufacturing techniques
	\end{ExpEntry}

	%NETWORK OPS, NCUSD
	\begin{ExpEntry}{Network Operations Intern, Naperville Community Unit School District 203}
		{June 2019}{Aug. 2019}
		\item Led a small team to conduct a suite of wireless surveys across the district
		\item Condensed findings into building level reports highlighting problems and providing solutions
		% \item Developed a Java application to analyze active survey data
		% \item Worked closely with both hardware and software infrastructure
	\end{ExpEntry} 

	%GOLF CADDIE NCC
	\begin{ExpEntry}{Golf Caddie, Naperville Country Club}
		{Apr. 2018}{Aug. 2020}
		\item Carries patrons clubs for duration of the golf round
		\item Engages directly with golfers to ensure an enjoyable and timely round
	\end{ExpEntry} 
\end{Experience}

%---------------------------------------------------------------------------Projects
%Similar to ``Skills,'' we put this section in two columns: prior and current projects
\begin{Projects}
	%Current projects
	\begin{TitleDescription}{Current Projects}
		\item Developing \href{https://github.com/Baricus/HaskTTP}{HaskTTP}, a conditionally compliant HTTP/1.1 webserver written in Haskell from the socket level
		\item Using Iris as a component of a VST specification for verifying C programs in Coq
	\end{TitleDescription}

	&
	%Prior projects
	\begin{TitleDescription}{Prior Projects}
		\item Writing a fully automated Twitch-specific IRC bot, \href{https://github.com/Baricus/BariBot}{BariBot}, in C++ from the socket level
		\item Explored elliptic curve arithmetic in Python over various fields
		\item Creating generative art through Java applications via the Processing3 library

		%\item Explored the arithmetic of elliptical curves through python-enabled computation, personally focused on creating classes representing points and the related abelian operations over various fields
		%\item Assembling and experimenting with the Open Theremin V3 Arduino Shield
		%\item Implemented a Spotify ``helper,'' managing volume, seek/skip, and liking/disliking songs through keyboard shortcuts via AutoHotKey
	\end{TitleDescription}	
\end{Projects}

%---------------------------------------------------------------------------Additional Activities
%For additional activities, each gets one ``bullet point''
\begin{Activities}
			\item Searching for the perfect macaroni and cheese recipe
			\item Amateur pianist
			\item Amateur poet looking to eventually publish
\end{Activities}
\end{document}
