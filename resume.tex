%=======================================
% resume.tex - a transition to typsetting in LaTeX
%
% Written by Miles Shamo
%	
%	My main goal in transitioning to LaTeX (LuaLaTeX techincally)
%	is to have better control over element layout, as I'm through
%	with dealing with MS Word textboxes.  At the same time,
%	I decided to upload this to github, to have better version control
%	for individual opportunities
%	
%	To Do:
%		- Redo Relevant coursework
%
%	Considerations:
%		- Should relevant coursework be revamped to entirely technical electives?
%		- Github Actions based releases rather than included PDF?
%		- Add vector work:
%			-bullets for education, skills, etc?


%------------------------------------------------------------------------------PACKAGES/SETUP
%page geometry setup
\documentclass[10.5pt]{resume}

% Header setup
\Author{Miles Shamo}
\Subtitle{Expected Bachelors and Masters of Science in Computer Science in 2023}
\Loc{Naperville, IL}
\Phone{630-465-8997}
\Email{milesshamo@gmail.com}

%---------------------------------------------------------------------------DOCUMENT
\begin{document}
%---------------------------------------------------------------------------Education
\begin{Education}
	\begin{EdEntry}{University of Illinois at Chicago, Chicago, Il.}{Aug. 2019}{May 2023}
		\item B.S. \& M.S. in Computer Science 
		\item 4.0 grade point average
		\item Honors College student
		\item Member of the A.C.M.
	\end{EdEntry}
\end{Education}

%---------------------------------------------------------------------------Coursework
\begin{Coursework}
		\begin{description}
			\item Data Structures (CS 251)
			\item Software Design (CS 342)
			\item Computer Algorithms (CS 401)
		\end{description} 
		
		&

		\begin{description}
			\item Formal Logic (MATH 430)
			\item Differential Equations (MATH 220)
			\item Linear Algebra (MATH 320)
		\end{description}
\end{Coursework}

%---------------------------------------------------------------------------Skills
\begin{Skills}
	%other skills
	\begin{description}
		\item Various Linux command line utilities
		\item Wireless networking analysis
		\item Regular expressions
		\item GNUMake, Valgrind, and GDB
		\item Computer maintenance and repair	
		%\item CAD software
	\end{description} 

	&

	%Programming languages
	\begin{TitleDescription}{Programming Languages}
		\item Exploring Haskell and Coq
		\item Experienced with C \& C++
		\item Proficient in Bash, Python \& \href{https://github.com/baricus/resume}{\LaTeX{}}
		\item Familiar with JavaScript, Java, AHK, \& Rockstar
	\end{TitleDescription} 
\end{Skills}

%---------------------------------------------------------------------------Experience (work)
\begin{Experience}
	% Telephony Studies CACI 2022
	\begin{ExpEntry}{Telephone Studies Intern, CACI}{May 2022}{Aug. 2022}
		\item Explored the effectiveness of honeypots as a method of collecting information with a focus on SSH attacks
		\item Analyzed binaries, network traffic, and other attacker activities to garner insights into individual attackers and trends
	\end{ExpEntry}

	% Telephony Studies CACI 2021
	\begin{ExpEntry}{Telephone Studies Intern, CACI}{May 2021}{Aug. 2021}
			\item Designed and developed a hypervisor-side anti-malware solution for guests utilizing virtual machine introspection
			\item Worked extensively with the Linux kernel to identify and test attack vectors
	\end{ExpEntry}

	%251 TA, UIC
	\begin{ExpEntry}{CS 251 (Data Structures) Teaching Assistant, University of Illinois at Chicago}
		{Aug. 2020}{May 2021}
			\item Engaged with students during self-run office hours, lab sections and as an oral exam proctor
			\item Provided iterative feedback on project design and requirements
	\end{ExpEntry}

	%MAKERSPACE, UIC
	\begin{ExpEntry}{Intern, University of Illinois at Chicago MakerSpace}
		{July 2020}{Aug. 2020}
		\item Independently designed and manufactured a desk-mounted microphone boom arm using MakerSpace facilities
		\item Designed and manufactured sneeze guards for use in UIC offices
		%\item Experimented with various additive and subtractive manufacturing techniques
	\end{ExpEntry}

	%NETWORK OPS, NCUSD
	\begin{ExpEntry}{Network Operations Intern, Naperville Community Unit School District 203}
		{June 2019}{Aug. 2019}
		\item Led a small team to conduct a suite of wireless surveys across the district
		\item Condensed findings into building level reports highlighting problems and providing solutions
		% \item Developed a Java application to analyze active survey data
		% \item Worked closely with both hardware and software infrastructure
	\end{ExpEntry} 

	%GOLF CADDIE NCC
	\begin{ExpEntry}{Golf Caddie, Naperville Country Club}
		{Apr. 2018}{Aug. 2020}
		\item Carries patrons clubs for duration of the golf round
		\item Engages directly with golfers to ensure an enjoyable and timely round
	\end{ExpEntry} 
\end{Experience}

%---------------------------------------------------------------------------Projects
%Similar to ``Skills,'' we put this section in two columns: prior and current projects
\begin{Projects}
	%Current projects
	\begin{TitleDescription}{Current Projects}
		\item Developing \href{https://github.com/Baricus/HaskTTP}{HaskTTP}, a conditionally compliant HTTP/1.1 webserver written in Haskell from the socket level
		\item Using Iris as a component of a VST specification for verifying C programs in Coq
	\end{TitleDescription}

	&
	%Prior projects
	\begin{TitleDescription}{Prior Projects}
		\item Writing a fully automated Twitch-specific IRC bot, \href{https://github.com/Baricus/BariBot}{BariBot}, in C++ from the socket level
		\item Explored elliptic curve arithmetic in Python over various fields
		\item Creating generative art through Java applications via the Processing3 library

		%\item Explored the arithmetic of elliptical curves through python-enabled computation, personally focused on creating classes representing points and the related abelian operations over various fields
		%\item Assembling and experimenting with the Open Theremin V3 Arduino Shield
		%\item Implemented a Spotify ``helper,'' managing volume, seek/skip, and liking/disliking songs through keyboard shortcuts via AutoHotKey
	\end{TitleDescription}	
\end{Projects}

%---------------------------------------------------------------------------Additional Activities
%For additional activities, each gets one ``bullet point''
\begin{Activities}
	\item Searching for the perfect macaroni and cheese recipe
	\item Amateur pianist
	\item Amateur poet looking to eventually publish
\end{Activities}
\end{document}
