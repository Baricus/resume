%=======================================
% resume.tex - a transition to typsetting in LaTeX
%
% Written by Miles Shamo
%	
%	My main goal in transitioning to LaTeX (LuaLaTeX techincally)
%	is to have better control over element layout, as I'm through
%	with dealing with MS Word textboxes.  At the same time,
%	I decided to upload this to github, to have better version control
%	for individual opportunities
%	
%	To Do:
%		- Redo Relevant coursework
%		- Add vector work:
%			-bullets for education, skills, etc?
%		- Convert to environments
%
%	Considerations:
%		- Should relevant coursework be revamped to entirely technical electives?
%		- Github Actions based releases rather than included PDF?


%------------------------------------------------------------------------------PACKAGES/SETUP
%page geometry setup
\documentclass[10.5pt, letterpaper]{article}
\usepackage[letterpaper, includeheadfoot, top=0.25in, bottom=0.25in, left=0.5in, right=0.5in, headheight=50pt, headsep=8pt]{geometry}

%font setup
\usepackage{fontspec}
\setmainfont[Ligatures=TeX]{Garamond}

%various other imports
\usepackage{fancyhdr}%headerst

\usepackage{xcolor}%color
\usepackage{hyperref}%hyperlink reference
\usepackage{microtype}%font resizing
\usepackage{tabularx}%for expandable columns
\usepackage{enumitem}%better lists
\usepackage{multicol}%multiple columns
\usepackage[compact]{titlesec} % allows setting title spacing


%hyperlink setup to better match standard
\hypersetup{final=true, colorlinks, urlcolor=blue}

%sets up rows for tabular eviornment
\setlength{\extrarowheight}{0pt}
\newcolumntype{E}{>{\raggedleft\arraybackslash}X}%Education columns (left side alignment)
\newcolumntype{Y}{>{\hsize=.11\hsize}X}%Year columns (fixed size, center aligned)

%tabularX
\renewcommand\tabularxcolumn[1]{p{#1}}
\renewcommand*{\arraystretch}{0}

%sets up descriptions
\setlist[description]{topsep=0pt, itemsep=2pt, parsep=0pt, style=nextline}

% vertical spacing setups
% lowes spacing of section headers (left spacing, above, below)
\titlespacing{\section}{0pt}{-10pt plus 0pt minus 2pt}{0pt plus 0pt minus 0pt}
% reduces multicol spacing
\setlength{\multicolsep}{0pt plus 0pt minus 0pt}
% reduces tabularx prespacing
\setlength{\textfloatsep}{0pt}
% removes space before front of lists
\setlength\topsep{0pt}
% lowers seperation between list items
\setlength\itemsep{3pt}

%Header setup
\pagestyle{fancy}
\fancyhf{}
\lhead
{
	{\Huge \bf Miles Shamo} \\
	{\Large \emph{Expected Bachelors of Science in Computer Science in 2023}}
}
\rhead 
{
	Naperville, IL \\
	630-465-8997 \\
\href{mailto:milesshamo@gmail.com}{milesshamo@gmail.com}}

%---------------------------------------------------------------------------DOCUMENT
\begin{document}
%---------------------------------------------------------------------------Education
%Formatted:
%	[School], city		elem3			start year
%		elem1			elem4			   --
%		elem2			elem5			end year
\section*{Education}

\begin{flushleft} %throws everything to the left to ensure table works properly
	\begin{tabularx}{\textwidth}[c]{E Y} %sets up table with 2 columns (ed and year)

		%UNIVERSITY OF ILLINOIS AT CHICAGO
		\begin{multicols}{2}
			\begin{description}
				\item [University of Illinois at Chicago, Chicago, Il.] 
					B.S. \& M.S. in Computer Science 
				\item 4.0 grade point average
				\item Honors College student
				\item Member of the A.C.M.
			\end{description}
		\end{multicols}
		%year
		&
		\begin{center}
			Aug. 2019 \\ to \\ May 2023
		\end{center}
		\\
	\end{tabularx}
\end{flushleft}

%---------------------------------------------------------------------------Coursework
\section*{Relevant Coursework}
\begin{flushleft}
	\begin{tabularx}{\textwidth}[c]{X X}
		\begin{description}
			\item Data Structures (CS 251)
			\item Software Design (CS 342)
			\item Computer Algorithms (CS 401)
		\end{description} &
		\begin{description}
			\item Formal Logic (MATH 430)
			\item Differential Equations (MATH 220)
			\item Linear Algebra (MATH 320)
		\end{description}
	\end{tabularx}
\end{flushleft}

%---------------------------------------------------------------------------Skills
%For skills, we can use a much simpler format
\section*{Skills}

\begin{flushleft}
	\begin{tabularx}{\textwidth}{X X}
		%other skills
		\begin{description}
			\item Various Linux command line utilities
			\item Wireless networking analysis
			\item Regular expressions
			\item GNUMake, Valgrind, and GDB
			\item Computer maintenance and repair	
			%\item CAD software
		\end{description} &

		%Programming languages
		\begin{description}
			\item [Programming Languages] 
				Exploring Haskell and Coq
			\item Experienced with C \& C++
			\item Proficient in Bash, Python \& \href{https://github.com/baricus/resume}{\LaTeX{}}
			\item Familiar with JavaScript, Java, AHK, \& Rockstar
		\end{description} 
	\end{tabularx}
\end{flushleft}

%---------------------------------------------------------------------------Experience (work)
%For this section, we return to the formatting of Education
\section*{Experience}

\begin{flushleft}
	\begin{tabularx}{\textwidth}[c]{E Y}
		% Telephony Studies CACI 2022
		\begin{description}
			\item [Telephone Studies Intern, CACI]
				Explored the effectiveness of honeypots as method of collecting information on internet attacks
			\item Analyzed binaries, network traffic, and other attacker activities to garner insights into individual attackers and trends
			
		\end{description}
		&
		\begin{center}
			May. 2022 \\ to \\ Aug. 2022
		\end{center}
		\\

		% Telephony Studies CACI 2021
		\begin{description}
			\item [Telephone Studies Intern, CACI]
				Designed and developed a hypervisor-side anti-malware for guests utilizing virtual machine introspection
			\item Worked extensively with the Linux kernel to identify and test attack vectors
		\end{description}
		&
		\begin{center}
			May. 2021 \\ to \\ Aug. 2021 
		\end{center}
		\\

		%251 TA, UIC
		\begin{description}
			\item [CS 251 (Data Structures) Teaching Assistant, University of Illinois at Chicago]
				Engaged with students during self-run office hours, lab sections and as an oral exam proctor
			\item Provided iterative feedback on project design and requirements
		\end{description}
		&
		\begin{center}
			Aug. 2020 \\ to \\ May 2021 
		\end{center}
		\\

		%MAKERSPACE, UIC
		\begin{description}
			\item [Intern, University of Illinois at Chicago MakerSpace]
				Independently designed and manufactured a self chosen project using MakerSpace facilities
			\item Designed and manufactured sneeze guards for use in UIC offices
				% \item Experimented with various additive and subtractive manufacturing techniques
		\end{description}
		&
		\begin{center}
			July 2020 \\ to \\ Aug. 2020
		\end{center}
		\\

		%NETWORK OPS, NCUSD
		\begin{description}
			\item [Network Operations Intern, Naperville Community Unit School District 203] 
				Led a small team to conduct a suite of wireless surveys across the district
			\item Condensed findings into building level reports highlighting problems and providing solutions
			% \item Developed a Java application to analyze active survey data
				% \item Worked closely with both hardware and software infrastructure
		\end{description} 
		& 
		\begin{center}
			June 2019 \\ to \\ Aug. 2019
		\end{center}
		\\

		%GOLF CADDIE NCC
		\begin{description}
			\item [Golf Caddie, Naperville Country Club] 
				Carries patrons clubs for duration of the golf round
			\item Engages directly with golfers to ensure an enjoyable and timely round
		\end{description} 
		&
		\begin{center}
			Apr. 2018 \\ to \\ Present
		\end{center} \\
	\end{tabularx}
\end{flushleft}

%---------------------------------------------------------------------------Projects
%Similar to ``Skills,'' we put this section in two columns: prior and current projects
\section*{Projects}

\begin{center}
	\begin{tabularx}{\textwidth}{X X}
		%Current projects
		\begin{description}
			\item [Current Projects] 
				Developing \href{https://github.com/Baricus/HaskTTP}{HaskTTP}, a conditionally compliant HTTP/1.1 webserver written in Haskell from the socket level
			\item Using Iris as a component of a VST specification for verifying C programs in Coq
		\end{description}

		&
		%Prior projects
		\begin{description}
			\item [Prior Projects] 
				Writing a fully automated Twitch-specific IRC bot, \href{https://github.com/Baricus/BariBot}{BariBot}, in C++ from the socket level
			\item Explored elliptic curve arithmetic in Python over various fields
			\item Creating generative art through Java applications via the Processing3 library

			%\item Explored the arithmetic of elliptical curves through python-enabled computation, personally focused on creating classes representing points and the related abelian operations over various fields
			%\item Assembling and experimenting with the Open Theremin V3 Arduino Shield
			%\item Implemented a Spotify ``helper,'' managing volume, seek/skip, and liking/disliking songs through keyboard shortcuts via AutoHotKey
		\end{description}	
	\end{tabularx}
\end{center}

%---------------------------------------------------------------------------Additional Activitiecls
%For additional activities, each gets one ``bullet point''
\section*{Additional Activities}

\begin{center}
	\begin{multicols}{2}
		\begin{description}
			
			\item Searching for the perfect macaroni and cheese recipe
			\item Amateur pianist
			\item Amateur poet looking to eventually publish
		\end{description}
	\end{multicols}
\end{center}


\end{document}
